% !TEX encoding = UTF-8 Unicode

%\documentclass[a4paper,10pt]{report}
\documentclass[a4paper,10pt,twocolumn]{report}
\usepackage[natbibapa,nosectionbib,tocbib,numberedbib]{apacite}
\AtBeginDocument{\renewcommand{\bibname}{literature}}
\usepackage{color,soul}

\usepackage[utf8x]{inputenc}
\usepackage{graphicx}
\usepackage{enumerate}
\usepackage{hyperref}


\usepackage[colorinlistoftodos]{todonotes}
\usepackage{pifont}

\usepackage{xcolor}
\newcommand\change[1]{\textcolor{red}{#1}}

\usepackage{lmodern}
\usepackage{listings}
\lstset{
	basicstyle=\scriptsize\ttfamily,
	columns=flexible,
	breaklines=true,
	numbers=left,
	%stepsize=1,
	numberstyle=\tiny,
	backgroundcolor=\color[rgb]{0.85,0.90,1}
}


\let\oldquote\quote
\let\endoldquote\endquote
\renewenvironment{quote}{\footnotesize\oldquote}{\endoldquote}

\title{Computational Communication Science 2\\ Course Manual}
\author{\emph{Course coordination and lectures:}\\ dr. Anne Kroon (a.c.kroon@uva.nl)\\dr. Marthe Möller (a.m.moller@uva.nl) \\ \\ \emph{Lab sessions:} \\Isa van Leeuwen (i.vanleeuwen@uva.nl) \\~\\College of Communication\\University of Amsterdam}
\date{Spring Semester 2023 (block 2)}


\begin{document}
	\maketitle
	
	\tableofcontents

	
	\chapter{About this course}
	
	This course manual contains general information, guidelines, rules and schedules for the course Computational Communication Science 2 (6 ECTs), part of the Communication in the Digital Society Minor offered by the College of Communication at the University of Amsterdam. Please make sure you read it carefully, as it  contains information regarding assignments, deadlines and grading.
	
	\section{Course description}
	
	A sales website that recommends new products to you personally, a company that uses a chatbot to answer your questions, or an algorithm that automatically identifies and warns you about fake news content: In our digital society, we use computational methods to communicate with each other every day. In this course, we will zoom in on the computational methods that lay behind these new ways of communication. We will explore the basic principles of their design, acquire an understanding of their implications, and learn how these methods can be used for science. We will work with some of these methods ourselves in the weekly lab sessions to get hands-on experience with these techniques and experience their advantages and limitations first hand. During weekly lectures, we will critically discuss the role that these methods play in our daily lives and what responsibility we have when working with them. At the end of the course, you will have a basic understanding of the methods that underlie different ways of communication in the digital society, you can formulate an informed opinion about the implications of these new techniques, and you will have some first-hand experience in working with them.
	
	In this 7-week course, each week consists of one lecture that zooms in on a specific computational method and the possible applications of this method and of one tutorial in which we work with this method. Through this mixture of introductions to computational methods in the lectures and a hands-on approach during the lab sessions, you will acquire knowledge on computational communication science that continues on the knowledge that you gained in CCS-2. In total, there are 28 contact hours in this course (7x 2-hour lecture and 7x 2-hour tutorial). 
	
	\section{Goals}
	Upon completion of this course, students should:
	\begin{enumerate}[a]
		\item Have a general understanding of state-of-the-art computational techniques useful to study communication phenomena in the digital society.
		\item Have a basic understanding of how to apply rule-based, unsupervised and supervised techniques to answer research questions in the field of communication science.
		\item Be able to identify key benefits and drawbacks associated with different rule-based and machine learning techniques.
		\item Have basic knowledge of what communication scientific questions can be answered using computational methods.
		\item Be able to apply a subset of these techniques independently in order to answer some basic research questions in the field of communication in the digital society.
		\item Have experience with independently solving problems in Python scripts by gathering information from online platforms.
		\item Be able to clearly communicate through written texts what steps were taken in a research project using computational methods.
	\end{enumerate}

	\section{Study materials}
	During this course, Python is the programming language that we will be working with. Hence, students should bring a laptop to each class, with a working Python environment installed.  
	
	In addition, a list of assigned readings is made available on the course Canvas page. All readings are available for download online using the UvA Digital Library or Google Scholar. If a reading is not available online, the material will be made available on the course Canvas page.
	
	
\chapter{Assignments, grading, and rules}
Assessment for this course is based on a mixture of individual and group assignments. 
	
	\section{Overview of assessments}
	The overall course grade is based on the following assignments:
	\begin{itemize}
		\item Regular multiple-choice questions (\(20\%\))
		\item Group assignment: Written report (\(20\%\))
		\item Group assignment: Presentation (\(10\%\))
		\item Take-home exam (\(50\%\))
	\end{itemize}

\section{Regular multiple-choice questions ($20\%$)}
At the start of the lab sessions, students will be asked to answer four questions about that week's literature and/or the preceeding lecture as well as about the content discussed in that week's lecture. All lab sessions will start with MC-questions, except for week 3 (19/04). This means that there are six weeks with MC-questions. In total, students can answer 24 questions (6 weeks * 4 questions). To receive full marks for this assignment, 18 questions need to be answered correctly. Hence, even if one of the lab sessions during which the MC-questions need to be answered is missed, it is still possible to receive full marks on the assignment. 
	
	\section{Group assignment: report ($20\%$)}
In groups of 4 students, you will work on a group assignment. See the description here in Section ~\ref{sec:groupassignment}
	
	\section{Group assignment: presentation ($10\%$)}
In week 7, you will present your group assignment in class. 
	
	\section{Take-home exam ($50\%$)}
	After the last lab session of the course (i.e., Tuesday, 24 May, 5 pm), students will receive a take-home exam. This exam will test students' understanding of concepts as discussed in the course. In addition, students are presented with a computational coding challenge on one of the techniques discussed in the course. The take-home exam is an open-book exam. Furthermore, it is an individual assignment that students make at home using their own computer. 

	\section{Grading}
	Students have to get a pass (5.5 or higher) for (1) the group report, and (2) for the take-home exam. If the grade of the group assignments or of the take-home exam is lower than 5.5, an improved version of the written work can be handed in within one week after the grade is communicated to the student(s). If the improved version still is graded lower than 5.5, the course cannot be completed. Improved versions of the group assignment and the take-home exam cannot be graded higher than 6.0. 
	
	\section{Deadlines and submitting assignments}
	Please send all assignments and papers as a PDF file to ensure that it can be read and is displayed the same way on any device. Hardcopies are not required. Multiple files should be compressed and handed in as one .zip file or .tar.gz file. Send your assignment via \url{https://filesender.surf.nl/} instead of direct email. Ensure that for all the files and/or folders you submit, your name(s) are included in the file-/ foldername. \\
	
	Assignments that are not completed on time, will be not be graded and receive the grade 1. 
		\begin{itemize}
			\item For the group report and the take-home exam, this means that all required files need to be submitted before the deadline. 
			\item The group presentation needs to held during the assigned class, meaning that slides or any other material used for the presentation needs to be present in class as well as at least one group member to give the presentation. 
		\end{itemize}
	Note that the deadline of an assignment is only met when the all files are submitted \emph{before} the deadline.

	\section{Plagiarism \& fraud}
	The provisions of the regulations governing fraud and plagiarism for UvA students apply in full to students of this course. The program Ouriginal will be used for the detection of plagiarism in written assignments. In submitting a text, the student consents to the text being entered in the database of the detection program concerned. Any suspicion of fraud and plagiarism (including self plagiarism) will be reported to the Examinations Board. To avoid doing prohibited things ‘by accident’ or ‘due to ignorance’, we urgently advise you to consult the following site about \textbf{fraud and plagiarism}: \url{https://student.uva.nl/en/content/az/plagiarism-and-fraud/plagiarism-and-fraud.html}
	
		\subsection{Attributing code}
		Plagiarism rules apply to code as well. It is understandable that your code is inspired by the code on the slides of the lectures, and by solutions found on sources such as \texttt{stackoverflow}. That is how coding works; you build on the insights and solutions of others. While it is perfectly fine to rely on solutions provided by others, it is \emph{highly important} that you are transparant with respect to \emph{where you got your code from}. More specifically, it should be clear what is your own code, and which parts are derived from other sources. That is, if you copy-paste code and use it in your group assignment or take-home exam, please include a reference to it's source (by simply including a comment like: \texttt{\# The following function is copied from https://stackoverflow.com/XXXXX/XXXXX} for (almost) literal copy-pasting, or \texttt{\# The code in this cell is inspired by https://...; I modified Y and Z}. 
	
	\section{Presence and participation}
	Attendance and active participation in the lectures and lab sessions are preconditions for obtaining the final grade. Students are expected to read the literature, to prepare and submit the assignments on time, and to actively participate in the discussions and exercises. \\

	Nonattendance is only allowed with a valid reason that is communicated to the lecturer before the meeting. If the student cannot attend a meeting, an email must be sent to the lecturer before the start of the session. Students are not allowed to miss more than two lectures or more than two lab sessions. When missing a meeting, students are still expected to submit all assignments on time. Students that skip more than two lectures or more than two lab sessions will not be able to complete the course. \\
	
	As you will have noticed in Computational Communication Science 1 and/or similar courses on coding, developing your skills to work with computational methods requires you to practice regularly and to be proactive when it comes to solving problems in your code. Hence, students of Computational Communication Science 2 are expected to practice with and revise the materials discussed in class at home. By practicing at home regularly in between classes, you will get the most out of this course and you will acquire the skills that you need to continue developing your programming knowledge after you finished this course. \\

	\underline{Class Lateness Policy:} Students are expected to arrive at class \underline{on time}. Being late twice will be considered as one nonattendance. \\

	\underline{Covid-19 Modifications:} Although this primary attendance policy stands, due to the Covid19 pandemic, any student with Covid or Covid-related symptoms must stay at home and take a test with the GGD.  Similarly, students are expected to follow all government guidelines related to quarantine. In these situations, students are permitted to livestream the lecture using the Zoom link established for the course. This livestream is considered attendance. \underline{Students must inform the teacher in advance} if they will be livestreaming the class. \\  

	\textbf{Note} that doctor's appointments, travel plans, bad weather, or other similar reasons are not sufficient reasons to follow the livestream – this attendance policy is specifically related to Covid. \underline{A maximum of three livestream sessions is allowed.} We also recommend that you find a buddy that can ask questions on your behalf during the session (livestream does not facilitate this well) and who can share any information you might miss.  If students must quarantine longer or are symptomatic longer than 3 livestreamed sessions, students must contact the study advisor. The study advisor can evaluate the situation and, as necessary, be in contact with the lecturer of the course or the Exam Committee depending upon the situation to discuss appropriate measures. 

	\section{Language}
	All meetings will be held in English. The assignments must be written and presented in English. 


	\section{Livestreaming of lectures and lab sessions}
	The livestreaming of lectures and lab sessions is \textbf{only} intended for students that have covid-19 symptoms or have to quarantine at home. 

	\section{Study advisors}
	The Study Advisers from Communication Science (https://student.uva.nl/communication-science/contact/study-advisers/study-advisers-cs.htm) are the go-to persons for all planning related questions concerning the Minor’s program and should be the ones contacted when indicated by the lecturers or by the course policies. They may refer you to the Study Advisers from your own course (if you do not follow the Communication Science Bachelor’s programme) when needed. 
	
	\section{Staying informed}
	It is your responsibility to check the means of communications used for this course (i.e., your email account, the course Canvas page, and the course Github page) on a regular basis, which in most cases means daily. 
	
	\section{Teaching team and contacting the lecturers}
	In this edition, the course will be taught by Anne Kroon, Marthe Möller, and Noon Abdulqadir. Anne and Marthe will teach the lectures and Noon will teach the tutorials.
	
	Please contact Anne \emph{and} Marthe for any questions related to the logistics of the lectures, assignments, or the course in general.
	Please contact Noon for any questions regarding the logistics of the tutorials. 
	
	Note that the team will not answer e-mails about specific coding issues (e.g., questions about error messages or general Python issues). Please use the tutorial meetings to ask question related to your specific code.
	

	\chapter{Group Assignment}
	\label{sec:groupassignment}
	This section describes the group assignment. 

\subsection*{Teams}

Form groups of around 4 students (5 is the maximum). Your tutorial teacher will help you with this.

\subsection*{Datasets}
Together with your group, you can select from one of the following datasets:

\begin{enumerate}
	\item \emph{Podcast dataset}:  \url{https://www.kaggle.com/datasets/roman6335/13000-itunes-podcasts-april-2018}
	\item \emph{News dataset}: \url{https://www.kaggle.com/datasets/szymonjanowski/internet-articles-data-with-users-engagement?select=articles_data.csv}
	\item \emph{Books dataset}: More specifically \texttt{google\_books\_1299.csv}: \url{https://www.kaggle.com/bilalyussef/google-books-dataset?select=google_books_1299.csv}. 
\end{enumerate}

When working with one of these datasets, it's crucial to note that not all columns need to be taken into consideration. You should choose the relevant variables yourself and provide justification in your assignment for including or excluding specific information.

The assignment will be centered around the choosen dataset. Your objective is to analyze the dataset in a data-scientific manner, and ultimately develop a recommender system.
The group assignment comprises three tasks: a written research report, code for exploring the dataset, and code for creating a simple knowledge-based or content-based recommender system. 

\section{Write a research report (30\% of final grade)}

The research report consists of...

\subsubsection{Method section}
\begin{itemize}
	\item Provide a detailed account of the steps you took, including the selection of variables, their types and how they were transformed. Describe your analytical strategy in detail.
	\item Describe the analytical strategy: i.e., the type of analytical strategy to be used to explore the dataset, such as visualizing the data, providing summary statistics with regards to variables, and/or using an inductive technique for text analysis, such as topic modelling to describe topics present in the data. Explain the techniques you plan to use to describe the dataset, and the reasons behind your choice of these techniques.
	\item Specify the type of recommender system you are developing and the rationale for choosing this type.
\end{itemize}

\subsubsection{Result section}
\begin{itemize}
	\item A description of the dataset (how many observations, what type of variables)
	\item Results of the explorative analysis (e.g., description of the topics you've found).
	\item Written demonstration of the recommender system: showcase the recommender system by providing an explanation of its workings, and demonstrating examples of the recommendations that can be generated for different types of input.
\end{itemize}

\section{Code for explorative data analysis (30\% of final grade)}

Explore, pre-process, and clean the dataset, and provide some descriptive analyses.

\begin{itemize}
	\item Explore the dataset, and inspect what type of relevant variables are present, what data can be used. Select which variables might be of interest and can be used later on.
	\item Feature engineering is an important step here (keeping in mind the type of descriptive analysis you want to conduct in step 2). 
	\item Describe the dataset using an inductive analysis.
	\item Provide a clear description of data you will be working with. For example, describe the most interesting variables in terms of data `type`, number of unique observations, mean, distribution, etc.
	\item Plotting the data, to visualise some of the relations in the dataset, is appreciated.
	\item Describe the dataset using some of the techniques as discussed in week 1 and week 3. For example, apply topic modelling techniques to describe topics present in the dataset.
\end{itemize}

\section{Code to create a recommender system. (20\% of final grade)}

Build a simple knowledge-based or content-based recommender system. 

\begin{itemize}
	\item Build a recommender system, based on the insights from week 4. It's up to you to decide whether you build a knowledge-based or content-based recommender system.
	\item Think about relevant features that you want to use in your algorithm design. Based on which features do you want to recommend content?
\end{itemize}

\section{Quality of the code and documentation (20\% of final grade)}

Make sure your code is well documented and understandable for people that see your code for the first time. 

\section{Handing in}
One member of your group can hand in the group assignment until Friday, 12 May, 17:00 via \url{https://filesender.surf.nl}. Send this to your tutorial teacher: Include i.vanleeuwen@uva.nl as recipient. \textbf{Please compress all files into a single .zip or .tar.gz file and use GroupAssignment as subject line.} 

Please include the following files:   
\section*{File formats}

\begin{itemize}
	\item  A set of scripts used to preprocess and analyse the data, and to build the recommender system
	\item \emph{
		You can hand in the answer to task 2 \emph{either} as \underline{one} Jupyter Notebook-file, integrating code, output, and explanations \emph{or} as one .py file containing the Python code and one PDF file with output and explanations.
	}
	\item The research report in .pdf format
\end{itemize}

\emph{Note: You may, but do \textbf{not} have to create a shared (public) github repository where you store all code and documentation. In that case, please include the link to the github repository together with the written assignment to your tutorial teacher.}

\textbf{Compress all files into one single .zip or .tar.gz file with your group name!}
If you want to compress your files on Linux, you can do so as follows. Imagine you have a folder called '/home/anne/groupassignment' in which you have everything you want to hand in, you can do

\begin{lstlisting}
cd /home/anne
tar -czf /home/anne/Desktop/groupassignment-team1.tar.gz groupassignment
\end{lstlisting}
to create a file \texttt{groupassignment-team1tar.gz} on your Desktop.

Then go to \url{https://filesender.surf.nl} and upload the file.
You get a mail that confirms that you have handed it. 

~\\
\textbf{\emph{Good luck!!!}}
	
	
	\chapter{Course Schedule}
	
	This course is set-up in the following manner. In a regular week, on the Mondays, we have lectures. Here, key concepts are explained from a conceptual and theoretical perspective, and examples of code implementations is python are provided.  On the Tuesdays, we have lab sessions. During the lab sessions, we generally start with knowledge questions about that week's literature as well as the preceeding lecture. Afterwards, students will work on assignments that are provided. When students have finished the assignments, they can work on the group assignment. The tuturial lecture (Isa) will walk around, and will help students with issues they encounter). If multiple students are encountering the same issues, these problems will be discussed plenarily. Active participation in the tutorial meetings is needed to ensure that students understand the materials, and can work on the (group) assignments.

\section*{Week 1: Course introduction \& Working with textual data}

\subsection*{Monday, 3--4: Lecture}
\textsc{\ding{52} Read this manual and inspect the course Canvas page.}\\
\textsc{\ding{52} Make sure your computer is ready to use for the course }\\
\textsc{\ding{52} Read in advance: \cite{Hirschenberg2015}.} \\
\textsc{\ding{52}Read in advance: chapter 10: Text as data \cite{van_atteveldt_computational_2022}.} \\
\textsc{\ding{52}Read in advance: \cite{Boumans2016}.} \\
During the first meeting, we will introduce the course: we explain the course goals and policies. We also take a first look into using computational methods for communication science by discussing what we can learn from analyzing texts.
We will discuss Bag-of-Words (BOW) representations of textual data, and discuss multiple ways of transforming text into matrices. 

\subsection*{Tuesday, 4--4. Lab session}
During our first tutorial lecture, we will take what we discussed in the lecture and use this to analyze text. We will start slow, and there will be time to go over some of the basic concepts as discussed in CCS-1.
We will practice with some simple top-down and bottom-up algorithm for text analysis. In order to prepare ourselves for more advanced bottom-up techniques in week 3, we will start practicing with transforming textual data to matrices, using  \texttt{sklearn}'s vectorizers. 

\section*{Week 2: Start of the group assignment }

\subsection*{Monday, 10--4: No lecture--Second Eastern Day}

\subsection*{Tuesdag, 11--4: Lab session}
During this tutorial meeting, we will form small groups of 3-4 students and start working on the group assignment. Specifically, we will make a start with exploring the dataset as provided, inspect relevant variables, and start working on cleaning the dataset and division of tasks. 

\section*{Week 3: Bottom up approaches to text analysis}

\subsection*{Monday, 17--4: Lecture}
%During this meeting, we will continue with were we left off in the first week. Using some of the techniques of week 1, we will explore some simple, deductive approaches and reflect on their advantages and disadvantes. As an alternative, we will reflect on some of the benefits of inductive, bottom-up approaches. To this aim, 
\textsc{\ding{52}Read in advance chapter 11: Automatic analysis of text \cite{van_atteveldt_computational_2022}.} \\
\textsc{\ding{52}Read in advance: \cite{Brinberg2021}.} \\

In this meeting, we will discuss some prominent techniques to analyses text using bottom up techniques. Specifically, we will discuss latent dirichlet allocation, a populair approach to detect topics in textual data. In addition, we will focus on different ways to measure similarity between texts, using cosine and soft cosine similarity. 

\subsection*{Tuesday 18--4: Lab session}
During this tutorial meeting, we will practice with the concepts and code as introduced in Monday's lecture. We will try to measure the similarity between sets of documents using different approaches. In addition, we will practice with a simple LDA model. \\
In case you understand all concepts, and finished the in-class assignments, you can continue working on the group assignment. 


\section*{Week 4: Recommender systems}

\subsection*{Monday, 24--4: Lecture}
\textsc{\ding{52} Read in advance: \cite{Moller2018}.}\\
\textsc{\ding{52} Read in advance: \cite{Loecherbach2020}.}\\

In today's lecture, we will discuss different types of recommender systems. In order to understand how recommender systems work, an understanding of the concepts as discussed in previous weeks is crucial. Specifically, in order to build a recommender system, one needs to understand how to preprocess data, transfrom textual data to data matrices, and calculate similarity between texts. 

\subsection*{Tuesdag, 25--4: Lab session}
During this week's lab session, we will practice with designing a recommender system yourself. You will experiment with different settings, and inspect the effects thereof on the recommendation content. 

\section*{Week 5: Taking a break}

\subsection*{Monday, 1--5: No Lecture -- UvA teaching free week}
\subsection*{Tuesday, 2--5: No Lab session -- UvA teaching free week}

\textsc{\ding{52} Take a break}\\

\emph{Note that this week is an "education-free week", meaning that there will not be a lecture or a tutorial this week. Take a well-deserved brake and we continue the course in week 6!}

\section*{Week 6: Text classification, part 1}

\subsection*{Monday, 8--5: Lecture}
\textsc{\ding{52} Watch: \url{https://www.youtube.com/watch?v=81vTqTz2pbM}.}\\
\textsc{\ding{52} Read in advance \cite{van_zoonen_social_2016}.}\\

In this lecture, we will take a look at text classification. We briefly review strictly rule-based methods and then we will explosure supervised machine learning.

\subsection*{Tuesday, 9--5: Lab session}
This week, you will present your group assignment. 

\subsection*{Deadline group project: Friday 12--5}
The deadline for handing in the group assignment is end of this week, \textbf{Friday 12--5 at 17:00}.

\section*{Week 7: Text classification, part 2}
\subsection*{Monday, 15--5: Lecture}
\textsc{\ding{52} Read in advance \cite{jordan_mitchell}.} \\
\textsc{\ding{52} Read in advance \cite{meppelink_reliable_2021}.}\\

In this lecture, we will take a deeper dive into supervised machine learning. We will discuss some commonly used machine learning models as well as how to validate classifiers.

\subsection*{Tuesday, 16--5: Lab session}
\change{This session is a hands-on approach to supervised machine learning.}

\section*{Week 8: Looking back and forward}

\subsection*{Monday, 22--5: Lecture}
\change{A critical look at Computational Methods for Communication Research --> beetje reflectie gaan we hier doen}

\subsection*{Tuesday, 23--5: Lecture}
\change{We will do something with a socratic seminar to discuss ethics and some philosophical stuff}. \\

\subsection*{Deadline take home exam: Friday 26--5}
The take-home exam will be made available on Wednesday, 24 May at 09:00. The deadline for submitting the take-home exam is \textbf{Friday 26 May at 17:00}.








	
	\bibliographystyle{apacite}
	\bibliography{literature.bib}
	

	
\end{document}