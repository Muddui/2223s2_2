This course is set-up in the following manner. In a regular week, on the Mondays, we have lectures. Here, key concepts are explained from a conceptual and theoretical perspective, and examples of code implementations is python are provided.  On the Tuesdays, we have lab sessions. During the lab sessions, we generally start with knowledge questions about that week's literature as well as the preceeding lecture. Afterwards, students will work on assignments that are provided. When students have finished the assignments, they can work on the group assignment. The tuturial lecture (Isa) will walk around, and will help students with issues they encounter). If multiple students are encountering the same issues, these problems will be discussed plenarily. Active participation in the tutorial meetings is needed to ensure that students understand the materials, and can work on the (group) assignments.

\section*{Week 1: Course introduction \& Working with textual data}

\subsection*{Monday, 3--4: Lecture}
\textsc{\ding{52} Read this manual and inspect the course Canvas page.}\\
\textsc{\ding{52} Make sure your computer is ready to use for the course }\\
\textsc{\ding{52} Read in advance: \cite{Hirschenberg2015}.} \\
\textsc{\ding{52}Read in advance: chapter 10: Text as data \cite{van_atteveldt_computational_2022}.} \\
\textsc{\ding{52}Read in advance: \cite{Boumans2016}.} \\
During the first meeting, we will introduce the course: we explain the course goals and policies. We also take a first look into using computational methods for communication science by discussing what we can learn from analyzing texts.
We will discuss Bag-of-Words (BOW) representations of textual data, and discuss multiple ways of transforming text into matrices. 

\subsection*{Tuesday, 4--4. Lab session}
During our first tutorial lecture, we will take what we discussed in the lecture and use this to analyze text. We will start slow, and there will be time to go over some of the basic concepts as discussed in CCS-1.
We will practice with some simple top-down and bottom-up algorithm for text analysis. In order to prepare ourselves for more advanced bottom-up techniques in week 3, we will start practicing with transforming textual data to matrices, using  \texttt{sklearn}'s vectorizers. 

\section*{Week 2: Start of the group assignment }

\subsection*{Monday, 10--4: No lecture--Second Eastern Day}

\subsection*{Tuesdag, 11--4: Lab session}
During this lab session, you will complete the first set of MC-questions. In addition, you will form small groups of 3-4 students and start working on the group assignment. Specifically, we will make a start with exploring the dataset as provided, inspect relevant variables, and start working on cleaning the dataset and division of tasks. 

\section*{Week 3: Bottom up approaches to text analysis}

\subsection*{Monday, 17--4: Lecture}
%During this meeting, we will continue with were we left off in the first week. Using some of the techniques of week 1, we will explore some simple, deductive approaches and reflect on their advantages and disadvantes. As an alternative, we will reflect on some of the benefits of inductive, bottom-up approaches. To this aim, 
\textsc{\ding{52}Read in advance chapter 11: Automatic analysis of text \cite{van_atteveldt_computational_2022}.} \\
\textsc{\ding{52}Read in advance: \cite{Brinberg2021}.} \\

In this meeting, we will discuss some prominent techniques to analyses text using bottom up techniques. Specifically, we will discuss latent dirichlet allocation, a populair approach to detect topics in textual data. In addition, we will focus on different ways to measure similarity between texts, using cosine and soft cosine similarity. 

\subsection*{Tuesday 18--4: Lab session}
During this tutorial meeting, we will practice with the concepts and code as introduced in Monday's lecture. We will try to measure the similarity between sets of documents using different approaches. In addition, we will practice with a simple LDA model. \\
In case you understand all concepts, and finished the in-class assignments, you can continue working on the group assignment. 


\section*{Week 4: Recommender systems}

\subsection*{Monday, 24--4: Lecture}
\textsc{\ding{52} Read in advance: \cite{Moller2018}.}\\
\textsc{\ding{52} Read in advance: \cite{Loecherbach2020}.}\\

In today's lecture, we will discuss different types of recommender systems. In order to understand how recommender systems work, an understanding of the concepts as discussed in previous weeks is crucial. Specifically, in order to build a recommender system, one needs to understand how to preprocess data, transfrom textual data to data matrices, and calculate similarity between texts. 

\subsection*{Tuesdag, 25--4: Lab session}
During this week's lab session, we will practice with designing a recommender system yourself. You will experiment with different settings, and inspect the effects thereof on the recommendation content. 

\section*{Week 5: Taking a break}

\subsection*{Monday, 1--5: No Lecture -- UvA teaching free week}
\subsection*{Tuesday, 2--5: No Lab session -- UvA teaching free week}

\textsc{\ding{52} Take a break}\\

\emph{Note that this week is an "education-free week", meaning that there will not be a lecture or a tutorial this week. Take a well-deserved brake and we continue the course in week 6!}

\section*{Week 6: Setting up supervised machine learning}

\subsection*{Monday, 8--5: Lecture}
\textsc{\ding{52} Watch: \url{https://www.youtube.com/watch?v=81vTqTz2pbM}.}\\
\textsc{\ding{52} Read in advance \cite{van_zoonen_social_2016}.}\\

This lecture is an introduction to supervised machine learning. We will discuss the basic principles of supervised machine learning using practical examples as well as example code. We will also discuss the role and applications of supervised machine learning in research and society.

\subsection*{Tuesday, 9--5: Lab session}
During this week's lab session, you will present your group assignment.

\subsection*{Deadline group project: Friday 12--5}
The deadline for handing in the group assignment is end of this week, \textbf{Friday 12--5 at 17:00}.

\section*{Week 7: Validation (in supervised machine learning)}
\subsection*{Monday, 15--5: Lecture}
\textsc{\ding{52} Read in advance \cite{jordan_mitchell}.} \\
\textsc{\ding{52} Read in advance \cite{meppelink_reliable_2021}.}\\

In this lecture, we will take a deeper dive into supervised machine learning whereby we also discuss how to validate classifiers.

\subsection*{Tuesday, 16--5: Lab session}
This week's lab session is a hands-on approached to supervised machine learning: you will set-up your own machine and train it to do some classifying. In addition, you will go through some approaches to validate your classifier.

\section*{Week 8: Coding in an academic context}

\subsection*{Monday, 22--5: Lecture}
\textsc{\ding{52} Read in advance \cite{baden_three_2022}.} \\

In the last lecture, we reflect on what we learned but mostly on how we can use this responsibly as scholars and researchers. 

\subsection*{Tuesday, 23--5: Lecture}
In the last lab session of the course, we will discuss scholars' use of code for research and the opportunities and responsibilities that this brings. It will also include a hands-on session on how to create responsible code.

\subsection*{Deadline take home exam: Friday 26--5}
The take-home exam will be made available on Wednesday, 24 May at 09:00. The deadline for submitting the take-home exam is \textbf{Friday 26 May at 17:00}.







