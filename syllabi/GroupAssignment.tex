This section describes the group assignment. 

\subsection*{Teams}

Form groups of around 4 students (5 is the maximum). Your tutorial teacher will help you with this.

\subsection*{Datasets}
Together with your group, you can select from one of the following datasets:

\begin{enumerate}
	\item \emph{Podcast dataset}:  \url{https://www.kaggle.com/datasets/roman6335/13000-itunes-podcasts-april-2018}
	\item \emph{News dataset}: \url{https://www.kaggle.com/datasets/szymonjanowski/internet-articles-data-with-users-engagement?select=articles_data.csv}
	\item \emph{Books dataset}: More specifically \texttt{google\_books\_1299.csv}: \url{https://www.kaggle.com/bilalyussef/google-books-dataset?select=google_books_1299.csv}. 
\end{enumerate}

When working with one of these datasets, it's crucial to note that not all columns need to be taken into consideration. You should choose the relevant variables yourself and provide justification in your assignment for including or excluding specific information.

The assignment will be centered around the choosen dataset. Your objective is to analyze the dataset in a data-scientific manner, and ultimately develop a recommender system.
The group assignment comprises three tasks: a written research report, code for exploring the dataset, and code for creating a simple knowledge-based or content-based recommender system. 

\section{Write a research report (30\% of final grade)}

The research report consists of...

\subsubsection{Method section}
\begin{itemize}
	\item Provide a detailed account of the steps you took, including the selection of variables, their types and how they were transformed. Describe your analytical strategy in detail.
	\item Describe the analytical strategy: i.e., the type of analytical strategy to be used to explore the dataset, such as visualizing the data, providing summary statistics with regards to variables, and/or using an inductive technique for text analysis, such as topic modelling to describe topics present in the data. Explain the techniques you plan to use to describe the dataset, and the reasons behind your choice of these techniques.
	\item Specify the type of recommender system you are developing and the rationale for choosing this type.
\end{itemize}

\subsubsection{Result section}
\begin{itemize}
	\item A description of the dataset (how many observations, what type of variables)
	\item Results of the explorative analysis (e.g., description of the topics you've found).
	\item Written demonstration of the recommender system: showcase the recommender system by providing an explanation of its workings, and demonstrating examples of the recommendations that can be generated for different types of input.
\end{itemize}

\section{Code for explorative data analysis (30\% of final grade)}

Explore, pre-process, and clean the dataset, and provide some descriptive analyses.

\begin{itemize}
	\item Explore the dataset, and inspect what type of relevant variables are present, what data can be used. Select which variables might be of interest and can be used later on.
	\item Feature engineering is an important step here (keeping in mind the type of descriptive analysis you want to conduct in step 2). 
	\item Describe the dataset using an inductive analysis.
	\item Provide a clear description of data you will be working with. For example, describe the most interesting variables in terms of data `type`, number of unique observations, mean, distribution, etc.
	\item Plotting the data, to visualise some of the relations in the dataset, is appreciated.
	\item Describe the dataset using some of the techniques as discussed in week 1 and week 3. For example, apply topic modelling techniques to describe topics present in the dataset.
\end{itemize}

\section{Code to create a recommender system (20\% of final grade)}

Build a simple knowledge-based or content-based recommender system. 

\begin{itemize}
	\item Build a recommender system, based on the insights from week 4. It's up to you to decide whether you build a knowledge-based or content-based recommender system.
	\item Think about relevant features that you want to use in your algorithm design. Based on which features do you want to recommend content?
\end{itemize}

\section{Quality of the code and documentation (20\% of final grade)}

Make sure your code is well documented and understandable for people that see your code for the first time. 

\section{Handing in}
One member of your group can hand in the group assignment until Friday, 12 May, 17:00 via \url{https://filesender.surf.nl}. Send this to your tutorial teacher: Include i.vanleeuwen@uva.nl as recipient. \textbf{Please compress all files into a single .zip or .tar.gz file and use GroupAssignment as subject line.} 

Please include the following files:   
\section*{File formats}

\begin{itemize}
	\item  A set of scripts used to preprocess and analyse the data, and to build the recommender system
	\item \emph{
		You can hand in the answer to task 2 \emph{either} as \underline{one} Jupyter Notebook-file, integrating code, output, and explanations \emph{or} as one .py file containing the Python code and one PDF file with output and explanations.
	}
	\item The research report in .pdf format
\end{itemize}

\emph{Note: You may, but do \textbf{not} have to create a shared (public) github repository where you store all code and documentation. In that case, please include the link to the github repository together with the written assignment to your tutorial teacher.}

\textbf{Compress all files into one single .zip or .tar.gz file with your group name!}
If you want to compress your files on Linux, you can do so as follows. Imagine you have a folder called '/home/anne/groupassignment' in which you have everything you want to hand in, you can do

\begin{lstlisting}
cd /home/anne
tar -czf /home/anne/Desktop/groupassignment-team1.tar.gz groupassignment
\end{lstlisting}
to create a file \texttt{groupassignment-team1tar.gz} on your Desktop.

Then go to \url{https://filesender.surf.nl} and upload the file.
You get a mail that confirms that you have handed it. 

~\\
\textbf{\emph{Good luck!!!}}